% From the Research Methods workbook, page 242.
%
% This section usually starts with a one- or two-line
% summary of the aim of the study and the main result or results. It
% compares the findings against the literature, and explains why
% some differences (if any) may have been found. This section
% sometimes contains the limitations of the study and suggestions for
% future work. It often ends with a conclusion.

This study explores whether guardrails, specifically retrieval-augmented generation (RAG), moderator models, and their combination, can ensure correctness in LLM-generated responses to medication-related medical prompts.
While all three guardrail setups improved upon the baseline, none achieved perfect correctness, highlighting both the promise and limitations of current guardrail strategies.

\subsection{Performance findings and trade-offs}
The results reveal a clear trade-off between precision and recall across the different guardrail mechanisms:

\begin{itemize}
\item Moderator-only achieved the highest recall (95.0\%), indicating it was most effective at identifying correct answers.
    However, it also had the highest false positive rate, suggesting a tendency to over-approve potentially unsafe or incorrect outputs.
\item RAG-only offered a more balanced performance, with moderate precision and recall (both 77.9\%), but still struggled with both false positives and false negatives.
\item Combined guardrails yielded the highest precision (78.8\%), meaning it was most conservative in approving only correct answers.
    However, this came at the cost of recall (74.3\%), as it missed a significant number of correct responses.
\end{itemize}
These findings suggest that while combining RAG and moderator models can reduce the risk of approving incorrect outputs, it may also lead to under-approving correct ones—an important consideration in clinical contexts where both false positives and false negatives carry risks.
This performance still includes a significant number of false positives and false negatives, up to the level where it is not safe to use the LLM-generated responses in a clinical setting without further human review.

\subsection{Inconsistent Model Behavior}
The confusion matrices and individual question breakdowns reveal inconsistencies in model behavior.
Notably, there were 12 cases where both RAG and Moderator approved an answer, but the Combined model did not.
This suggests that the sequential application of guardrails may yield unexpected differences in outcomes, making the model unreliable in certain scenarios.

Conversely, in 4 cases, the Combined model approved answers that RAG alone had rejected.
Again this highlights unexpected behvaior in the models, where same prompts can lead to different outcomes.

Causes of the inconsistencies may include:
\begin{itemize}
    \item Temperature settings, which influence the randomness of the model's responses.
    \item Top-p sampling, which controls the diversity of the generated text.
\end{itemize}
Further research is needed to understand how these parameters affect model behavior and output stability.

\subsection{Correctness Criteria and Operationalization}
Correctness was operationalized based on expert-reviewed criteria: appropriate medication, dosage, and indication tailored to patient context.
This strict definition ensured clinical relevance but may have penalized otherwise reasonable alternatives.
Future work could explore graded correctness or confidence scoring to better reflect real-world clinical decision-making, where multiple valid options may exist.
The introduction of reasoning and cofidence scoring has been explored in \autoref{kang2024r}.
A similar approach could be applied to the guardrail approaches in this study to better reflect the clinical decision-making process.

\subsection{Input Quality and Model Limitations}
The quality of the input prompts significantly influenced model performance.
During the manual review phase the expert reviewer indentified that some medical questions were incomplete or relevant patient information was missing.
This highlights the importance of high-quality, well-structured prompts in eliciting accurate model responses.
Further improvement of the performannce cna be achieved by introducing input guardrails to ensure that the input is up to a level where reasonalble answers can be expected.
For interactive applications the input guardrails can be used to guide the user to provide the necessary information.
This will make sure that the input passes only when a certain level of input quality confidence is reached.
An implementation that mimics the interaction of a medical professional with a patient is described in \autoref{hsu2025medplan}.

\subsection*{Conclusion}
This research demonstrates that guardrails can significantly improve the correctness of LLM outputs in medical contexts, but no single researched approach guarantees perfection.
The choice of guardrail depends on the desired balance between safety (precision) and inclusivity (recall).
A combined approach offers the most conservative filtering, but may miss valid answers.
Future research should focus on refining guardrail logic, exploring input validation, and testing across diverse guardrail models to build more trustworthy AI systems for healthcare.
