% From the Research Methods workbook, page 242.
%
% This section usually starts with a one- or two-line
% summary of the aim of the study and the main result or results. It
% compares the findings against the literature, and explains why
% some differences (if any) may have been found. This section
% sometimes contains the limitations of the study and suggestions for
% future work. It often ends with a conclusion.

ToDo: Investigate some samples of false positives and false negatives in the different guardrail approaches.

ToDo/Future: Are there instable answers in different runs of the same guardrail approach? Are there parameters that influence the stability of the answers (Temp/topP)?
% https://cloud.google.com/vertex-ai/generative-ai/docs/learn/prompts/adjust-parameter-values#:~:text=Gemini%20models%20support%20a%20temperature,a%20default%20temperature%20of%201.0.

Further research is needed to understand the trade-offs between different guardrail approaches and their effectiveness in ensuring output correctness.
Further research on use of input guardrails is needed to ensure that the input to the LLM is correct and relevant.